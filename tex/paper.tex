%!TEX root = youth_draft.tex - wont compile by itself
\sectionL{Introduction} \label{intro}

Policy preferences typically vary across the life cycle: young voters
may care more about education, employment, and climate, while older
voters may care more about fiscal and socio-cultural issues. Since
electoral politics the world over is largely the preserve of middle
aged men\footnote{the global median age of legislators is 53 which is
well over the median age of 29
\parencite{union2012global,united2015world}}, one might reasonably
suspect that public policy may not be representative of the
preferences of all voters. This problem is particularly acute in the
developing world, where the median age is typically even lower, but
the political class is often dynastic and geriatric\footnote{We
provide summary statistics on the age representation gap in India in
appendix \ref{sub:agerepIndia}}. Lowering the voting age has been
proposed as a policy to improve the representation of youth interests
by generating electoral pressure on political parties to compete over
young voters \parencite{vote16WP}. In this paper, we estimate the
effects the largest recorded youth enfranchisement, the lowering of
the voting age from 21 to 18 in India in the late 1980s, and find
negligible electoral effects.

Compositional changes in the electorate, such as through women's
suffrage \parencite{Miller2008-xn,Teele2018-vz,Morgan-Collins2021-yq},
introduction of compulsory voting \parencite{fowler2013electoral}, or
voting technology that serves as a de-facto enfranchisement of a
subset of the electorate
\parencite{hidalgo2012renovating,fujiwaraVotingTechnologyPolitical2015,desai2019technology},
have been shown to have substantial political and policy consequences,
typically in favor of the newly enfranchised group. However, these
political consequences are contingent on the newly enfranchised group
exercising their newly granted right to vote. Little evidence exists
on whether newly-enfranchised youth exercise this right. Answering
this question satisfactorily is challenging because these reforms are
simultaneously implemented at the national level, therefore leaving us
with no obvious comparison units, while across-country comparisons are
rife with omitted-variables bias. The scant work that attempts to
study the consequences of lower voting ages compares national level
turnout before and after the reduction
\parencite{mcallister2014politics} and is likely confounded by
aggregate trends in turnout, which were steadily decreasing in OECD
countries in the 20th century. Yet, despite the scant evidence on
electoral consequences of voting-age changes, there is policy-interest
in voting-age changes, as exemplified by the failed amendment to lower
the voting age to 16 in HR1 \parencite{vote16USAHR1} in the US, as
well as similar movements across multiple OECD countries.

In this paper, we study the political consequences of a large
franchise extension to the youth -- the lowering of the voting age
from 21 to 18 -- implemented by the 61st amendment of India in
December 1988 \parencite{sharma2021introduction}. In the late 1980s,
then prime minister Rajiv Gandhi spearheaded reforms to lower the
voting age from 21 to 18, and consequently added nearly 50 million new
eligible voters \parencite{Ani_2019,Pachauri1989-oc}, which was nearly
7\% of the population or 13.5\% of the voting age
population\footnote{This figure is based on the population in 1985,
which was 784 million \parencite{desa2015united}, and the voting age
population ($\geq 21$, which was $\approx$ 380 million
\parencite{Pachauri1989-oc})}, which constitutes the largest franchise
extension since Indian independence and democratisation in 1950. We
use variation in the age-composition of constituencies at the time of
the reform to conduct difference-in-differences comparisons before and
after the implementation of the amendment in 1989 to estimate the
effects of the youth franchise extension on voter turnout, incumbent
politicians' re-election rates, and electoral competition in state
level politics. We find that the policy led to no statistically
detectable changes in the size of the registered electorate, turnout,
incumbency, and political competition, and if anything, marginally
lowered turnout rates and competition.

These findings are in stark contrast to the expectations of the policy
in the Indian news media at the time \parencite{Pachauri1989-oc},
where there was widespread optimism about youth participation and the
anticipated increase in electoral competition thanks to the sheer size
of the youth vote. Our finding that a unprecedented injection of young
voters into the electorate had negligible political consequences
suggests that legal changes alone are unlikely to improve youth
representation in politics. This finding also offers a potential
explanation for the persistence of this stark gap: parties correctly
anticipate low participation from the youth in electoral politics, and
as such tailor their electoral strategies to older voters. This
low-representation state is sustained by and feeds into low youth
entry into politics.

Our findings contribute to the extensive literature in political
economy on the causes and consequences of franchise extensions
\parencite{acemogluTheoryPoliticalTransitions2001,aidt2013democratization,aidt2015democratization,berlinski2011political}.
Much of this literature focusses on reforms in the 1800s that
enfranchised adult men, and as such the consequences of such franchise
extensions are typically studied with an eye towards redistribution
and finds mixed evidence for the theoretical prediction that
redistribution should rise following franchise extensions
\parencite{Meltzer1981-gh}. Studies on recent de-jure and de-facto
franchise extensions of the poor
\parencite{fujiwaraVotingTechnologyPolitical2015,cassan2020enfranchisement},
find substantial electoral and public policy effects. Studies
focussing on womens' suffrage in the US
\parencite{Morgan-Collins2021-yq} point the pivotal role of suffrage
movement strength in converting de-jure changes into de-facto
representation of political interests; our findings are consistent
with this theory and represent a negative case where the absence of a
social movement to coordinate newly enfranchised young voters may
explain the negligible electoral consequences of the policy.

Our findings also contribute to the literature on causes and consequences
of low youth participation and representation in politics. Scholars
typically point to supply-side explanations for low participation such
as low political ambition among young people
\parencite{lawless2015running} and restrictive minimum age
requirements. Some recent work, such as \textcite{mcclean2021does} and
\textcite{curry2018lawmaker}, also examines the consequences of youth
under-representation and finds that electing older politicians affects
welfare spending and redistributive policies more generally, which are
marred by intergenerational conflict. Yet, much existing work is
missing an explanation for why political parties choose not to tailor
campaigns and nominate candidates to turn out a large and potentially
pivotal portion of the electorate. By focussing on a particular
episode of enormous youth enfranchisement, we study a `best-case'
setting for youth representation, and find that even here, the
franchise extension had negligible electoral consequences.

The rest of the paper is organized as follows:
\ref{sec:research_design} describes the data and research design,
\ref{sec:res} reports results from the difference-in-differences and
event study analyzes and examines potential mechanisms, and
\ref{sec:conc} concludes.


% subsection  (end)

% ########     ###    ########    ###
% ##     ##   ## ##      ##      ## ##
% ##     ##  ##   ##     ##     ##   ##
% ##     ## ##     ##    ##    ##     ##
% ##     ## #########    ##    #########
% ##     ## ##     ##    ##    ##     ##
% ########  ##     ##    ##    ##     ##


\sectionL{Data and Design} % (fold)
\label{sec:research_design}

\subsection{Data} % (fold)
\label{sub:data}

\subsubsection{Treatment}
Like most franchise extensions, the franchise extension applied to all
elections held after 1988. In order to evaluate the effects of the
policy, we use the age composition of different constituencies to
generate variation in the 'intensity of treatment' of the policy.
Constituencies with relatively younger populations were naturally
affected more by the franchise extension.

To construct this treatment intensity measure, we need granular data
on age composition of sub-national administrative units. To our
knowledge, age composition is only available at the state level until
the 1981 census. However, since we're interested in effects on
state-legislature elections, this is insufficiently granular. We
therefore use the 1991 census, which was first Indian census that
reports age composition and cohort-level education at the district
level. We define `treatment' as having youth share above the
state-median.  While these are nominally measured post-treatment
(1991, a year after the first elections with lower voting age) voter
registration is sufficiently protracted in India to rule out any large
short run changes in political competition because of migration. This
measure is valid under the assumption that there was not much
migration from `treated' (younger) to `control' (older) constituencies
in response to the policy change, which we believe is likely. We
report the spatial distribution of age in 1991, as well as our coding
of binary treatment status using the median as the cutoff, in figure
\ref{fig:map_panels}. We find a fair amount of variation in the age
composition of constituencies within and across states; while some of
the densest states in the gangetic plains (Uttar Pradesh, Bihar,
Jharkhand) lean older, while Deccan and Southern states have a mix
young and old constituencies. In our preferred
difference-in-differences specification, we use \emph{within-}state
variation in age-composition relative to state medians.

% \bland
\begin{figure}[tb]
  \centering
  \includegraphics[]{../output/figs/treatment_map2.png}
  \caption{Spatial distribution of youth share quantiles (left) and
  treatment discretised at state-median (right). District level data from
  1991 census. Bottom panel plots density of the youth share median}
  \label{fig:map_panels}
\end{figure}
% \eland

\subsubsection{Electoral outcomes}
We restrict our analysis to state-assembly (\emph{Vidhan-Sabha})
instead of also analysing national-parliamentary (\emph{Lok-Sabha})
elections three main reasons. First, since Lok Sabha constituencies
much larger and aggregate multiple districts, the across-constituency
variation in `treatment intensity' quite small;
\textcite{Pachauri1989-oc} estimates that the amendment enfranchised
nearly 100,000 new voters in each of the 545 Lok-Sabha constituencies.
Secondly, the number of Lok-Sabha constituencies within each state is
much smaller than the number of Vidhan-Sabha constituencies, thereby
making within-state difference-in-differences necessitated by our
preferred specification intractably noisy. Third, because the two
parliamentary elections immediately following the amendment, the
parliamentary elections of 1989 and 1991, were marred by controversy
and the latter was called off-cycle following the dissolution of
parliament, which makes it unsuitable to evaluate the dynamic effects
of the policy. Furthermore, the 1991 election was marred by
low-turnout in the initial phase followed by a surge in turnout
prompted by the assassination of Rajiv Gandhi between the two rounds
of the election \parencite{blakeslee2018politics}. Since these shocks
pull constituencies in opposite directions, and since the preceding
events altered the stakes of the election substantially beyond the
nominal increase in the electorate from the amendment, we deem the
first two post-period hopelessly contaminated for a reasonable
difference-in-differences comparison for Lok-Sabha elections.

To prepare the analysis sample, we spatially merge districts level age
composition summaries to assembly-constituency shape-files for the 3rd
delimitation \parencite{infomap_vs_maps}. Assembly constituencies are
typically but not always wholly contained inside districts
\footnote{The median district contains 7 assembly constituencies. When
assembly constituencies overlap more than 1 district (beyond merge
error, which we set to be 1\% area), we use areal weights to aggregate
the two districts' age composition to an AC level one. }. We then
merge the assembly constituency level treatment measure to
assembly-constituency level electoral data from
\textcite{Jensenius2017-qf}, who collate all elections at the
parliamentary (Lok-Sabha) and state parliamentary (Vidhan-Sabha) level
since 1960. This allows us to construct a panel of assembly
constituencies from 1975 onwards (since electoral boundaries were
redrawn then) with registered voter numbers, turnout, incumbent
re-election and vote share, effective number of parties, and winning
margin. We report the aggregate time series (in red) and separate
trends for treatment and control areas for the six political outcomes
of interest in fig \ref{fig:agg_trends}. The two groups appear to be
trending in tandem for most variables before and after the
implementation of the 61st amendment.


\begin{figure}[tb]
  \centering
  \includegraphics[]{../output/figs/agg_ts.pdf}
  \caption{Aggregate trends in political outcomes by constituency
  type. Dotted line delineates the pre-amendment and post-amendment
  periods}
  \label{fig:agg_trends}
\end{figure}


\subsection{Research Design} % (fold)
\label{ssub:research_design}

Since the treatment applies everywhere after 1988, we propose using a
difference-in-differences style comparison between places `more' and
`less' affected by the franchise extension to estimate the effects of
the policy. The estimand of interest, unlike in conventional DiD
settings, is the difference in treatment effects (individually
identified using a pre-post comparison conditional on fixed-effects
and time trends) rather than the ATT.

We begin with a nonparametric first-differences regression of the form

$$
\Delta y_i = f( \text{youth share}_i) + \epsi_i
$$

where we regress changes in political outcomes $\Delta y_i$ between
the last pre-amendment and first post-amendment election on the share
of young voters in constituency $i$. We estimate this function
nonparametrically using local linear regression (and overlay a linear
fit) in order to avoid functional form assumptions on the effect of
youth share on political outcomes. This specification also helps
evaluate potential non-linearities in treatment effect as a function
of youth share. The consistency of this estimation strategy relies on
the assumption of uniform-moderation which stipulates that treatment
effects are monotonic in the moderator (age-composition of
constituencies).

We then use a difference-in-differences design to compare turnout and
political competition in constituencies before and after the
implementation of the 61st amendment. States hold elections every five
years, but operate on different cycles\footnote{We report the last
pre-period and first post-period election in table~\ref{tab:elec_pre_last}.
However, since we standardise this into event time, this is not a
\emph{staggered} difference in differences, which is a design rife
with problems \parencite{goodman2018difference}}. We standardise these
elections into event-time relative to the first post-amendment
election (0), and estimate fixed-effects regressions of the form

\begin{align}
Y_{ijt} & = \alpha_i + \gamma_t + \tau D_{ijt}  + \epsi_{ijt} \label{eqn:twfe} \\
Y_{ijt} & = \alpha_i + \psi_{jt} + \tau D_{ijt} + \epsi_{ijt} \label{eqn:styfe}
\end{align}

where $i$ indexes constituencies, $j$ indexes states, and $t$ indexes
time, with $\alpha_i, \gamma_t, \psi_{jt}$, and  denoting constituency
, election, and state $\times$ election fixed effects. $D_{ijt}$ is
the `treatment' dummy, which takes on a value of 1 for youth
constituencies (i.e. constituencies with above-median youth share in
the state, which is time invariant because we only observe it for one
cross-section) after the amendment passed. This is akin to a standard
difference-in-differences regression, where the `treated' and `post'
dummies are included in the constituency- and time FEs. Since the
`treatment' only varies at the district level, we deem this the level
of treatment assignment and cluster standard errors at the district
level throughout, which is more conservative than clustering by the
panel unit (constituency level).



For ~\ref{eqn:twfe} to yield consistent estimates, we need parallel
trends across states for political outcomes. The particular parallel
trends is somewhat non-standard, since in this case it demands
parallel trends between two treated groups with differential
intensity. Since political outcomes are typically a function of
state-level party politics and policy changes, we believe that the
general parallel trends assumption is likely implausible. To address
this concern, we use specification \ref{eqn:styfe}, which adds state
$\times$ election fixed-effects. This restricts comparisons the
\emph{within-state} variation in political outcomes, and so accounts
for many potential time-varying state level confounders. This means
that we estimate difference-in-differences between younger and older
constituencies \emph{within each state}, which connects cleanly with
our `treatment' definition. The distinction between \ref{eqn:twfe} and
\ref{eqn:styfe} amounts to whether we are comparing outcomes in the
green and magenta regions in fig \ref{fig:map_panels} throughout the
whole country (as in \ref{eqn:twfe}) or within each state boundaries
(as in \ref{eqn:styfe}). We consider parallel trends to be more likely
to hold in \ref{eqn:styfe}, and as such consider this our preferred
specification. We also estimate event-study regressions that decompose
the treatment effect over time (relative to the period immediately
preceding the amendment, $t = -1$).




% subsubsection research_design (end)
% subsection data (end)
% section research_design (end)

%%%%%%%%%%%%%%%%%%%%%%%%%%%%%%%%%%%%%%%%%%%%%%%%%%%%%%%%%%%%%%
% ########  ########  ######  ##     ## ##       ########  ######
% ##     ## ##       ##    ## ##     ## ##          ##    ##    ##
% ##     ## ##       ##       ##     ## ##          ##    ##
% ########  ######    ######  ##     ## ##          ##     ######
% ##   ##   ##             ## ##     ## ##          ##          ##
% ##    ##  ##       ##    ## ##     ## ##          ##    ##    ##
% ##     ## ########  ######   #######  ########    ##     ######

\sectionL{Results}
\label{sec:res}


\subsection{First-differences with continuous treatment} % (fold)
\label{sub:first_differences_estimates}

We begin by using a simple first-differences regression where we
regress changes in political outcomes on constituency youth-share in
1990. In figure \ref{fig:continuous_treatment}, we plot the
distribution of first-differences as a function of continuous
variation in the `treatment' - the share of the population in the
franchise-extension age group, both residualised on state-fixed
effects (which is equivalent to our preferred State $\times$ time
fixed effects specification in the two-period setting).We find very
little non-linearity across all outcomes, and consistently find
precisely estimated null effects (reported as the slope from the
linear regression in each panel)\footnote{To evaluate the robustness
of this approach, we replicate the same figure but with
first-differences between last two pre-amendment election outcomes and
report it in \ref{fig:continuous_treatment_placebo}. We find
negligible effects in the placebo. }.

\begin{figure}[tb]
  \centering
  \includegraphics[width=0.9\textwidth]{../output/figs/scatter_outcomes_prepost.pdf}
  \caption{First differences in electoral outcomes plotted against
  standardised youth share. Differences are computed between levels in
  the first post-amendment election minus the last pre-amendment
  election (i.e. $t \in \SetB{-1, 0}$, and residualised on state
  fixed-effects. We report a linear LOESS smoother and the linear
  regression coefficient from the continuous treatment below each
  panel.}
  \label{fig:continuous_treatment}
\end{figure}

% subsection first_differences_estimates (end)

\subsection{Fixed-effects regression estimates}

Next, we report results from estimating the different
regressions specifications for log number of registered voters and log
turnout count (voters who turned out) \footnote{ We choose to work
with the number of voters turning out (the numerator in turnout rates)
as opposed to computed turnout rates because the latter is a ratio
with $\# \text{electors}$ as the denominator, and interpreting it
requires assumptions on how much the denominator grows relative to the
numerator. Working with the numerator alone simplifies the
interpretation.} in table \ref{tab:regres_turnout}. Both these
coefficients are to be interpreted as a percent-change. We find that
the effect of youth-franchise on log-number of voters  is effectively
zero to the third decimal point in our preferred specification (column
2). Similarly, we estimate a small effect of youth franchise on
turnout on the order of 1 percentage point, although this is also
statistically indistinguishable from 0.

These results can be thought of as an ecological `first stage' effect
of the youth franchise, which is necessary for any potential effects
on political outcomes. Since we fail to find that the size of the
electorate and turnout rates meaningfully changed in response to the
enfranchisement, we may anticipate that there were few, if any,
downstream effects. Since these outcomes are aggregated for the entire
electorate, the standard ecological inference problem applies : we
don't strictly observe youth registration and turnout and therefore
only indirectly test for their magnitudes holding registration and
turnout rates in the rest of the electorate constant. While a large
displacement of other segments of the electorate by newly enfranchised
youth voters is also consistent with our findings, we find such an
magnitude implausible in the setting under consideration.

\input{../output/tables/turnout_regs.tex}

We then turn to studying whether incumbents were voted-for and
re-elected at higher rates (conditional on running, hence fewer
observations). We find that in our preferred specifications, the
confidence interval for the effect covers zero and is mildly positive,
suggesting that the effect was likely zero.

\input{../output/tables/incumb_regs.tex}

Finally, we examine whether the introduction of young voters altered
political competition by studying the effects on the Effective Number
of Parties (the inverse Hirschman-Herfindahl index of vote
shares), and the winning margin of the winning candidate. In both
cases, we find that the effects were very small or zero. The ENOP
results are notable in how much they differ between the basic two-way
specification and the within-state specification, suggesting that
changes in political competition are largely based on uniform swings
by state.

\input{../output/tables/competition_regs.tex}


\subsection{Event Study estimates} % (fold)
\label{sub:event_study_results}

We now proceed to decompose treatment effect dynamics using an event
study regression, thereby both verifying the plausibility of parallel
trends, as well as examining treatment effect dynamics using the basic
specification, one with state-level time trends, and state$\times$year
fixed effects (our preferred specification) in fig
\ref{fig:evstudy_main}. For the most part, we find parallel trends is
plausible most specifications, since the lead-estimates ($t = -2$) are
generally zero. Estimates are also remarkably stable across
specifications, with the state $\times$ year FE specification yielding
the most precise estimates. Decomposing the effects over time also
allows us to address the potential concern that negligible effects in
the first election immediately following the amendment (denoted by 0
in our event study figures) may have been driven by lack of
information among the newly enfranchised, and that effects would
appear in subsequent elections.

We begin with the event study results for number of registered voters
and voter turnout counts in row 1. For the log-voters outcome we find
that confidence intervals for all specifications cover zero and rule
out large growth in the electorate in response to the franchise
extension. In the latter, we find that if anything, turnout fell,
although these effects are small and appear with a lag. Next, we
examine incumbent re-election and vote share in row 2, and again find
that the estimates cover zero and rule out meaningful effects.
Finally, we study the effects on ENOP and winning margin in row 3. We
find that estimates are generally small and statistically
indistinguishable from zero. Again, we rule out meaningful effects for
both outcomes.

\begin{figure}[tb]
  \centering
  \includegraphics[width=\textwidth]{../output/figs/coefplot_evstudy.pdf}
\caption{Event Study: Log Registered Voters, Log Turnout, Incumbent
  Reelected (binary), Incumbent vote share, ENOP, and winning margin}
\label{fig:evstudy_main}
\end{figure}
\newpage

\subsection{Heterogeneity} % (fold)
\label{ssub:mechanisms}

We now examine whether the null effects of franchise extension on
political outcomes are different for constituencies with high and low
levels of youth education, since one of the motivations for the policy
was to specifically empower the educated youth of India. To evaluate
this, we use fully moderated interactions with bins of constituency
illiteracy rate and education rates (defined as the share of the youth
population with no literacy and secondary-school education
respectively) as suggested by \textcite{hainmueller2019much}.

We report these results graphically in figures
\ref{fig:het_te_electors_edu}. We find that the treatment effect
appears to not vary significantly with either youth illiteracy of
youth higher-education rates (binned into high / medium / low). The
binned estimates are largely consistent with a simple linear
interaction (indicated by the black line in the figures), and as such
we fail to detect substantial heterogeneity in these effects.


Next, we examine treatment effect heterogeneity by constituency `type'
in fig \ref{fig:het_te_constype} by estimating the event-study
separately for the three types of constituencies. India's system of
electoral reservations ensures that certain constituencies in both the
national parliament (Lok Sabha) and state parliaments (Vidhan Sabha)
are reserved for candidates from ST and SC groups on the basis of
population shares. SC and ST were experiencing relatively high
population growth at the time \parencite{kulkarni2005population}, so
one has reason to believe that electoral effects would be largest in
these constituencies. However, we detect very little heterogeneity in
most outcomes with the exception of winning margin, where we find that
SC constituencies became less competitive over time, possibly due to
incumbent entrenchment.

% subsubsection mechanisms (end)
\subsection{Robustness Checks}

\subsubsection{Panel Estimators} % (fold)
Conventional two-way fixed effects regressions estimates are typically
inconsistent for the Average Treatment effect on the Treated (ATT)
estimand under treatment effect heterogeneity
\parencite{SantAnna2020-ly,Imai2020-dg}. While the problem is most
severe in designs that use staggered treatment adoption (which is not
the case for our setting), some estimators proposed to address the
treatment heterogeneity problem in panel data settings have the added
benefit of typically performing pre-treatment matching, which
alleviates concerns regarding the parallel trends assumption by
matching treated units with control units with similar outcome
trajectories. This is in the spirit of the synthetic control method
\parencite{abadie2010synthetic}, although our setting is unsuitable
for use of synthetic control methods since we have only two
pre-treatment time periods.

We therefore use the panel-matching estimator
\parencite{imai2019should} that explicitly matches on
trajectories even for short panels and also allows us to exact-match
on covariates. This effectively estimates the treatment effects using
the subset of units for which the parallel trends assumption holds. To
mimic our preferred specification, we exact-match on state, which
effectively restricts the pool of matches for any treated constituency
to other constituencies within that state, thereby holding many
potential time-varying confounders constant. We report the
panel-matching estimates for our six outcomes in fig
\ref{fig:panelmatch}, and find similar results to the regression
specification and event study. One notable difference is that we the
negative turnout effects are now statistically significant.

\subsubsection{Recoding the treatment} In the main analysis, we define the
treatment threshold at the median, which mechanically means that we
assign similar units to treatment and control depending on whether
they cross this arbitrary threshold. As an alternative, to maximise
the contrast between the two groups, we restrict our analysis to
constituencies that are below the 25th percentile in youth share
(which we call control) or exceed the 75th percentile in youth share
(which we call treatment). This yields a `maximum-contrast' version of
the analysis sample where treatment and control groups are more
meaningfully different in ex-ante youth share. We then re-estimate the
treatment effect using our preferred specification (eqn
\ref{eqn:styfe}) and report the estimates alongside the primary
estimates from the corresponding specifications (from tables 1-3) in
\ref{fig:p2575_compare}. Similarly, we re-estimate the event-study
specifications and report the two sets of coefficients in
\ref{fig:p2575_compare_evstudy}. The point estimates for turnout are
somewhat larger and more negative, but overall, the estimates are
effectively identical and (mechanically) more noisily estimated.

%  ######   #######  ##    ##  ######  ##       ##     ## ########  ########
% ##    ## ##     ## ###   ## ##    ## ##       ##     ## ##     ## ##
% ##       ##     ## ####  ## ##       ##       ##     ## ##     ## ##
% ##       ##     ## ## ## ## ##       ##       ##     ## ##     ## ######
% ##       ##     ## ##  #### ##       ##       ##     ## ##     ## ##
% ##    ## ##     ## ##   ### ##    ## ##       ##     ## ##     ## ##
%  ######   #######  ##    ##  ######  ########  #######  ########  ########


\sectionL{Conclusion}
\label{sec:conc}


We estimate the electoral effects of the extension of the franchise to
18-21 year olds in India in the late 1980s and find that the extension
had negligible electoral consequences. This precise null electoral
effect of an unprecedented number of young voters offers pessimistic
predictions regarding the effectiveness of lower voting ages as a
means of improving youth representation. These findings also provide a
potential explanation for why political parties continue to cater
campaigns and electoral messaging to older voters - they correctly
anticipate that there are limited electoral penalties for doing so.
This result also suggests what likely \emph{won't} work to increase
the substantive representation of the youth - simply giving them the
vote is insufficient, since they seemingly don't use it. Increasing
youth representation and engagement in politics likely demands a
different, more creative, set of policies.

We suggest that the lack of electoral effects may be driven by the
absence of organizations that coordinate newly enfranchised voters,
which was a key source of the effects of women's suffrage in the US
\parencite{Morgan-Collins2021-yq}. Future work in contexts with
age-disaggregated turnout and preference data may be able to uncover
these mechanisms directly by estimating cohort-specific turnout
effects, as well as differences in political preferences across age
cohorts.


